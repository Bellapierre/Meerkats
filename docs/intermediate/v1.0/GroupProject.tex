\documentclass{article}

\usepackage{array}

\title{Meerkats File Synchronizer - Initial Report}
\author{Made in GB}
\date{}
\begin{document}
\maketitle{}


\section{Project Description}
% \subsection{Project Description}

File synchronization has grown increasingly important as enterprises use the cloud as a means for collaboration. It is also commonly used for backup and for mobile access to files. In this project, our team is collaborating to develop file synchronization software. We have chosen to name our software Meerkats, because the meerkats are one of the most collaborative animals in the world, and one of the main feature for our software is collaboration.

\subsection{Project Goals}
The goals of the project are divided into two categories, technical goals and management goals.

\subsection{Technical Goals}

\begin{enumerate}
  \item \textbf{Develop File Synchronisation Software} File synchronization software is used to store copies of files to another device or to the cloud. The files are typically available to be accessed via a Web-based portal. Some examples of file sync software include Box and Dropbox.
  \item \textbf{Develop two clients (desktop, mobile)} The desktop will be built for the windows OS, and the mobile application will be using the android operating system. Clients manipulate files within a diretory according to the commands from server.
  \item \textbf{Build a server} Central server receives requests from clients and responds with commands to clients which indicates different local manipulation.

\end{enumerate}

\subsection{Management Goals}

\begin{enumerate}
  \item \textbf{Develop Communication Skills} Good communication skills are the most basic skills that one can possess as an employee or student. We aim to improve the team members ability to communicate effectively with each others and  to convey information in a simple and unambiguous way.
  \item \textbf{Practice Diversity} The team consists of 6 members who come from different countries, speak different languages and have different backgrounds. We aim to learn how to recognize individual differences and understand how cultural differences can impact how people work, and interact.
  \item \textbf{Project Management Skills} Take this project as an opportunity to build project management skills that are essential to successfully complete a project that includes but not limited to planning, leadership, communication, risk management.
  \item \textbf{Teamwork Skills} Teamwork is important for the success of this project. We aim to build teamwork skills which are essentials at work after graduation. Each member will learn how to be a good team player by demonstrating skills such as negotiation, communication, problem solving and prioritization.
\end{enumerate}

\subsection{Implementation Strategy}
The team will follow the waterfall model throughout the project. To comply with the project requirements, the team selected development model to be branch/pull request model based on git. The team agreed to choose golang for the server side and C sharp for the desktop client as well as Java for android client.

\subsection{Project Deliverables}
By completing this project, the following deliverables are achieved.

\begin{enumerate}

  \item \textbf{File Storage Server} This component is used to store files and deal with requests received from users.
  \item \textbf{Mobile Client} It is an Android solution. Users can use the app on the go and will enjoy features such as Sign Up, Log In, View Files, Upload Files, Rename Files, Delete Files, etc.
  \item \textbf{Desktop GUI Client} The desktop GUI client is based on the Windows OS and it includes the same features as the mobile client.
  \item \textbf{Meerkats File Synchronizer Report} This report provides detailed information about the project in terms of team members, project plan, deliverables, timeline, technologies used, challenges and much more. It acts as a central reference for users who are interested in the software.
  \item \textbf{Meerkats File Synchronizer Presentation} The team will put together slides to present the work done. The team will demonstrate the solution and its layout, features and will walk the audience through the functionalities supplied by the software.

\end{enumerate}

\subsection{Implementation Timeframe}
Below table outlines the milestones along with brief description and expected timeframe.

\begin{center}
\begin{tabular}{ | m{3em} | m{2cm}| m{3cm} | m{1cm} | m{1cm} | m{1cm} |}
\hline
\textbf{\#} & \textbf{Milestones} & \textbf{Description} & \textbf{Start Date} & \textbf{End Date} & \textbf{Status}  \\
\hline
1 & Kick-off meeting & Background survey and initial plan & Jan 21, 2019 & Jan 27, 2019 & cell32 \\
\hline
2 & Software Design  & Design the software layout & Jan 28, 2019 & Feb 3, 2019 & cell33 \\
\hline
3 & Software Development & Work to deliver the file synchronization software & Feb 4, 2019 & March 21, 2019 & cell3 \\
\hline
4 & Initial Report & To develop and submit the initial report & Jan 29, 2019 & Feb 7, 2019 & cell32 \\
\hline
5 & Group Presentation & To demonstrate  the group initial software design and share the project plan & Feb 8, 2019 & Feb 8, 2019 & cell33 \\
\hline
6 & Software Testing &  To test the software and ensure correct and secured implementation & March 15, 2019 & March 21, 2019 & cell3 \\
\hline
7 & Final report & To submit the final report describing complete information about the software and the project lifecycle & Feb 8, 2019 & March 28, 2019 & cell32 \\
\hline
8 & Final Presentation & To deliver the final presentation about the software & March 22, 2019 & March 29, 2019 & cell33 \\
\hline
\end{tabular}
\end{center}

\subsection{Project Progress}
After two meetings with the group, it was discovered that the language could be a barrier toward successful implementation of the project. A solution to tackle this challenge was to make sure that we speak clearly and use simple communication and ask confirming questions such as is that clear to everyone? as this is important to avoid misunderstanding and make sure that every on the same page for the every aspect of the project.

\section{Project Organization}

% \subsection{Project Organization}
This project requires each member to play different roles and carry different responsibilities throughout the project lifetime. The following roles and responsibilities are stipulated below and have been agreed by the team

\begin{enumerate}

  \item \textbf{System Developers} All team members participate in developing the software. Each individual develops a piece of the software. The developers follow specific software development guidelines to ensure consistency among the solution parts to allow efficient and smooth integration and high performance.
  \item \textbf{System Testers} Each team member will develop a test scenario and will ask the other team members to execute it. All test scenarios will be devised and reviewed by the team members during the group project update meeting and prior to the testing phase.
  \item \textbf{System Documenters} Two members will build the project reports (initial and final). Inputs to the reports will be submitted by each member and according to the current stage of the project implementation plan.
  \item \textbf{Report Proofreading} Upon the completion of the final report. The team will proofreading the report and share feedback prior to the final submission.

\end{enumerate}

\textbf{Responsibility:}

\begin{itemize}
  \item \textbf{Boyang Zhang:} Server-side Development /Integration Test / Unit Test
  \item \textbf{Xi He:} Android Client-side Develop / Integration Test / Unit Test
  \item \textbf{YiFeng Zheng:} Android Client-side Development / Unit Test
  \item \textbf{Yenan Huang:} Desktop Client-side Development / Unit Test
  \item \textbf{Frida Solheim:} Initial Report and presentation / Desktop development / Final Report / Unit Test
  \item \textbf{Samah Alghamdi:} Initial Report and presentation / Android development / Final Report / Unit Test
\end{itemize}

\subsection{Collaboration Tools} The team will be using Google Docs to share links and important updates, GIT to store project documentations and slack to instantly communicate information regarding the project.

\subsection{Peer Assessment} It was agreed by everyone that the final points should be divided equally by the team members. In the unlikely event of low commitment of one or more team members, the distribution of the points will be discussed again in a special meeting with everyone.

\subsection{Conflict Management}
The team members are inspired to provide friendly project environment that enables everyone to put the best efforts as they can as well as to build resilient team that respond to challenges, unforeseen events and different circumstances in timely manner and with the ability to continue meeting the planned dates and delivering quality outcomes.

One way to avoid conflict is to use consensus for important decisions and issues. For less important issues, we will rely on the subject matter expert with input from others.

Additionally, to ensure a successful outcome, the team has setup communication ground rules that benefit everyone and are effective should conflict arise. It has been agreed to fully abide by those rules and to review them frequently to ensure they make sense to everyone.

\subsection{Communication Ground Rules}
\begin{enumerate}
  \item Mutually commit to our team’s objectives as stated in the project report or negotiate until we can make this mutual commitment.
  \item All team members are expected to attend team meetings unless they are out of town or sick. If a team member is unavailable, he or she should notify the rest of the team and should share their update through email.
  \item Team meetings will start and end on time.
  \item Action items will be distributed within 24 hours after the meeting.
  \item Understand each other’s styles.
  \item Tackle issues, not people.
  \item Permit one speaker at a time (avoid side conversations).
  \item Bring issues to the table during the team update meeting.
  \item Explain the reasoning leading to your conclusions.
  \item Invite inquiry into your views.
  \item Inquire into the reasoning of others.
\end{enumerate}


\end{document}
